\documentclass[9pt, twoside]{book}

% Packages
\usepackage{graphicx}
\usepackage{book-of-common-prayer}
\usepackage{titlesec}
\usepackage{Zallman,lettrine}
\usepackage{parskip}
\usepackage{microtype}
\usepackage{ragged2e}
\usepackage{fancyhdr}
\usepackage{enumitem}

% Settings
\clubpenalty = 10000
\widowpenalty = 10000 
\displaywidowpenalty = 10000
\geometry{
  a6paper,
  left=12.5mm, 
  right=12.5mm, 
  top=12.5mm, 
  bottom=12.5mm
}
\pagestyle{headings}
\fontspec{PTSerif-Regular.ttf}
\setcounter{secnumdepth}{-1}
\emergencystretch=3em
\renewcommand\LettrineFontHook{\Zallmanfamily}
\titleformat{\chapter}{\Large\scshape\filcenter}{\thechapter}{}{}
\titleformat{\section}{\large\scshape\filcenter}{\thesection}{}{}
\titleformat{\subsection}{\scshape\filcenter}{\thesubsection}{}{}
\titlespacing*{\chapter}{0pt}{14pt}{14pt}
\titlespacing*{\section}{0pt}{10pt}{10pt}
\titlespacing*{\subsection}{0pt}{6pt}{6pt}
\fancyhead{}
\fancyfoot{}
\renewcommand{\headrulewidth}{0pt}
\fancyhead[RO,LE]{\thepage}
\fancyhead[CE]{\leftmark}
\fancyhead[CO]{\rightmark}
\pagestyle{fancy}
\setlength{\parindent}{15pt}
\setlist[enumerate]{labelsep=3pt,leftmargin=15pt}

% Title
\title{\textsc{Auxilium Christianorum}}
\author{}
\date{}

% Document
\begin{document}

\maketitle
\thispagestyle{empty}

\chapter{Exordium}
\thispagestyle{empty}

\vfill
\lettrine{E}{cclesia} nos docet quod ipsa in Ecclesia Triumphans (complectens membra Ecclesiæ quæ sunt in cœlo), Sufferens (complectens membra Ecclesiæ quæ sunt in purgatorio) et Militans (intimans ista membra quæ in hoc mundo versantur) divisa est. Cum enim pars Ecclesiæ Militantis essemus, certemus in bello spirituali et hoc deposcit nos ut agnoscamus, sicut Apostolus docuit, “Quoniam non est nobis colluctatio adversus carnem et sanguinem, sed adversus principes, et potestates, adversus mundi rectores tenebrarum harum, contra spiritualia nequitiæ, in cælestibus." \textit{(Eph. 6:12)}.\par
Oportet autem Membra Auxilii Christianorum ut semper realitatem status eorum prout membra Ecclesiae viventium in mundo in animo habeant. Adquin sententia Apostoli, plurimi Catholici non munus suum ad bellum gerendum contra hostes demonicos serio suscipiuntur, propterea enim Sodalitas Auxilii Christianorum condita est.

\clearpage

\noindent Sodalitatis fines principales hujus sint:
\begin{enumerate}
    \item Ad preces providendas sacerdotibus adscitis cum Auxilio Christianorum quin apostolatus eorum efficax sit in expellendis dæmonibus.
    \item Ad preces ornandas ut sacerdotes, membra Sodalitatis familiæque custodiantur et non inutiliter adficiantur à dæmonibus.
\end{enumerate}

\section{Obligationes Membrorum}

Obligationes membrorum Auxilii Christianorum sunt:
\begin{enumerate}
    \item Antequam fiunt membra Auxilii Christianorum, laici enixe monentur ut se ad confessoremvel rectorem spiritualem conferant.
    \item Necesse est eis agere vitam habitualem gratiæ sanctificantis, numquam velint in peccatum letale labi mortalem et semper vitent omne peccandi propositum venialem.
    \item Jugiter, oportet membra petere incrementum et perfectionem orationis habitualis; quæ amplectitur non solum preces vocales quæ infra continentur et obligatoriæ sunt, sed etiam vitam crebrae constantis meditationis; cum valeat ad detrudendam demonicam et vitandum maleficium dæmonicum.
    \item Orare Rosarium cotidianum. Intentio Rosarii qualiscumque sit et non necesse est ei offerri ad fines Auxilii Christianorum.
    \item Oportet membra implére obligationes cotidianas precium vocalium quæ infra continentur cum finibus principalibus hujus Sodalitatatis pro intentione. Oportet membra frequentare sacramentalia arcendi potestate expellendique dæmones nota.
    \item Membra certent servare verba Apostoli in cordibus suis cum pugnemus contra principalitates potentiasque, id est contra dæmonicam, certent esse mitia et humilia ex parte proximi sui, et numquam adflictent ex irâ et ultione, sed petant convellere opem dæmonicam secundum statum vitæ suæ. Hoc continet usum precium ligantium secundum principia authentica Catholica et vitationem ullæ superstitionis num sit in versante cotidianâ vel in gerendo bellum contra spiritus malignos. Hoc significat usum precium infra contentorum aut aliæ precis ad dæmonicam expellendam semper debeant subire principia authentica Catholica et fiant cum devotione fideque.
    \item \textls[-10]{Membra debeant certare augére devotionem suam Beatæ Mariæ Virgini sub titulo Virginis Potentis.}
    \item Singuli debeant certent augére devotionem suam custodi Angelo sui.
    \item \textls[-4]{Cum possibilis sit ex parte ærarii, omnia membra Auxilii Christianorum debent obtinére signa Beatæ Virginis Mariæ et Sanctus Michælis pro domu sua, quibus cereas votivas ferventes antepositas.}
    \item \textls[-18]{Membra Sodalitatis debeant tam certiorari ut agnoscant nullam obligationium ligat poenâ peccati.}
\end{enumerate}

\noindent Cum approbatione ecclesiasticâ.\\
Copyright MMXVII Auxilium Christianorum.

\chapter{Preces Cotidianæ Oblatæ}
\thispagestyle{empty}

\section{Preces sunt cotidie dicendae}

\begin{vresponses}
\V{Audjutorium \cross nostrum in nomine Domini.}
\R{Qui fécit cælum et terram.}
\end{vresponses}

\lettrine{O}\ \ Piissima Virgo Maria, quæ caput serpentis contrivisti, protege nos a vindicta mali. Offerimus tibi dolores, bona, operaque ut ea purifices, sanctifices et largiaris Filio tuo sicut oblationem perfectam. Hæc oblatio fit ne dæmonia qui afficere membra Auxilii Christianorum petunt cognoscant originem expulsionis et cæcitatis suae. Cæca eos ne nostra opera bona cognoscant. Cæca eos ne cognoscant quos ulcantur. Cæca eos ut sententiam iustam operum suorum suscipiant. Operi nos sanguine pretioso Filii tui ut protectione quæ ab Passione Morteque ejus fluit fruamur. Amen.

\textsc{Sancte} Michæl Archangele, defende nos in proelio, contra nequitiam et insidias diaboli esto præsidium. Imperet illi Deus, supplices deprecamur: tuque, Princeps militiæ cælestis, Satanam aliosque spiritus malignos, qui ad perditionem animarum pervagantur in mundo, divina virtute, in infernum detrude. Amen.

\textsc{Angele} Dei, qui custos es mei, Me tibi commissum pietate superna; Hac nocte (die) illumina, custodi, rege, et guberna. Amen.

\textls[-17]{\textsc{Pater Noster}, qui es in cælis, sanctificetur nomen tuum. Adveniat regnum tuum. Fiat voluntas tua, sicut in cælo et in terra. Panem nostrum quotidianum da nobis hodie, et dimitte nobis debita nostra sicut et nos dimittimus debitoribus nostris. Et ne nos inducas in tentationem, sed libera nos a malo. Amen.}

\textsc{Ave Maria}, gratia plena, Dominus tecum. Benedicta tu in mulieribus, et benedictus fructus ventris tui, Iesus. Sancta Maria, Mater Dei, ora pro nobis peccatoribus, nunc, et in hora mortis nostræ. Amen.

\textsc{Gloria} Patri, et Filio, et Spiritui Sancto. Sicut erat in principio, et nunc, et semper, et in sæcula sæculorum. Amen

\subsection{Litaniæ Pretiosissimi Sanguinis \\ Domini Nostri Iesu Christi}

{\RaggedRight
Kyrie, eleison.\\
Christe, eleison.\\
Kyrie, eleison.\\
Christe, audi nos.\\
Christe, exaudi nos.\\
Pater de cælis, Deus, \textit{miserere nobis.}\\
Fili, Redemptor mundi, Deus, \textit{miserere nobis.}\\
Spiritus Sancte, Deus, \textit{miserere nobis.}\\
Sancta Trinitas, unus Deus, \textit{miserere nobis.}\\
Sanguis Christi, Unigeniti Patris æterni,\\\tab \textit{salva nos.}\\
Sanguis Christi, Verbi Dei incarnati,\\\tab \textit{salva nos.}\\
Sanguis Christi, Novi et Æterni Testamenti,\\\tab \textit{salva nos.}\\
Sanguis Christi, in agonia decurrens in terram,\\\tab \textit{salva nos.}\\
Sanguis Christi, in flagellatione profluens,\\\tab \textit{salva nos.}\\
Sanguis Christi, in coronatione spinarum emanans,\\\tab \textit{salva nos.}\\
Sanguis Christi, in Cruce effusus,\\\tab \textit{salva nos.}\\
Sanguis Christi, pretium nostræ salutis,\\\tab \textit{salva nos.}\\
Sanguis Christi, sine quo non fit remissio,\\\tab \textit{salva nos.}\\
Sanguis Christi, in Eucharistia potus et \\ lavacrum animarum,\\\tab \textit{salva nos.}\\
Sanguis Christi, flumen misericordiæ,\\\tab \textit{salva nos.}\\
Sanguis Christi, victor dæmonum,\\\tab \textit{salva nos.}\\
Sanguis Christi, fortitudo martyrum,\\\tab \textit{salva nos.}\\
Sanguis Christi, virtus confessorum,\\\tab \textit{salva nos.}\\
Sanguis Christi, germinans virgines,\\\tab \textit{salva nos.}\\
Sanguis Christi, robur periclitantium,\\\tab \textit{salva nos.}\\
Sanguis Christi, levamen laborantium,\\\tab \textit{salva nos.}\\
Sanguis Christi, in fletu solatium,\\\tab \textit{salva nos.}\\
Sanguis Christi, spes poenitentium,\\\tab \textit{salva nos.}\\
Sanguis Christi, solamen morientium,\\\tab \textit{salva nos.}\\
Sanguis Christi, pax et dulcedo cordium,\\\tab \textit{salva nos.}\\
Sanguis Christi, pignus vitæ æternæ,\\\tab \textit{salva nos.}\\
Sanguis Christi, animas liberans de lacu Purgatorii,\\\tab \textit{salva nos.}\\
Sanguis Christi, omni gloria et honore dignissimus,\\\tab \textit{salva nos.}\\
Agnus Dei, qui tollis peccata mundi,\\\tab \textit{parce nobis, Domine.}\\
Agnus Dei, qui tollis peccata mundi,\\\tab \textit{exaudi nos, Domine.}\\
Agnus Dei, qui tollis peccata mundi,\\\tab \textit{miserere nobis, Domine.}
}

\begin{vresponses}
    \V{Redimisti nos, Domine, in sanguine tuo.}
    \R{Et fecisti nos Deo nostro regnum}
\end{vresponses}
\vfill

\noindent Oremus.
Omnipotens sempiterne Deus, qui unigenitum Filium tuum mundi Redemptorem constituisti, ac eius sanguine placari voluisti: concede, quæsumus, salutis nostræ pretium ita venerari, atque a præsentis vitæ malis eius virtute defendi in terris, ut fructu perpetuo lætemur in cælis. Per eundem Christum Dominum nostrum. Amen. 

\vfill
\section{Dominica}

\lettrine{O}\ \ Gloriosa Regina Cœli et terræ, Virgo Potens, quæ habuisti contere caput serpentis antiqui calcaneo tuo, veni et utere hac potentiâ ab gratiâ Immaculatæ Conceptionis tuæ fluente. Tege nos sub amictu puritatis et dilectionis tuæ, trahe nos in dulcem domicilliam cordis tui; dele atque rende hostes impotentes ecfatos adnullare nos. Veni, Superana Domina Sanctorum Angelorum et Domina Sacratissimæ Rosariæ, quæ ab initio recepisti potentiam et missionem à Deo ad caput Satanæ conterendum. Emitte, supplices quæsumus, sanctas legiones tuas ut te imperante et cum potentiâ tuâ persequantur spiritus malignos, undique compescant, repelle audaces impetus eorum et procul a nobis expellant eos, nocentes nulli in via, ligantes eos juxta pedem Crucis judicari et sententiam ferri ab Jesu Christo, Filio tuo atque disponi ad libitum suum.

\clearpage

Sancte Joseph, Patronus Universalis Ecclesiæ, veni fer nobis auxilium contra potentias tenebrarum, repelle impetús diaboli et libera membra Auxilii Christianorum, et eos quibus sacerdotes ejusdem Sodalitatis orarent, à robore hostis.

Sancte Michæl, invoca totam cohortem coelestem ut commitant potentias suas in hoc prœlio feroce contra vires inferi. O Custodes Angeli, dirigite et protegite nos. Amen.

\vfill
\section{Feria II}

\lettrine{D}{omine} Jesu Christe, oramus ut tu tegeas nos, familias, et omnes
possessiones nostras cum dilectione et Pretiosissimo Sanguine tuo et
circumdes nos cum Angelis cælestis, Sanctis et amictu Benedictæ Matris
Nostræ. Amen. 

\vfill
\section{Feria III}

\lettrine{D}{omine} Jesu Christe, quæsumus te pro gratiâ ut remaneamus tuti in tutela
amictús Mariæ, circumdati cum rubo sancto ex quo sancta corona spinarum
facta est et intincta est Pretiosissimo Sanguine Tuo et Angelis Custodibus in
potentia Sancti Spiritus, ad majorem Patris gloriam. Amen.

\section{Feria IV}
\lettrine{I}{n} Nomine Jesu Christi, Domini et Dei nostri, Te deprecamur efficere omnes
spiritús esse impotentes, paralyticos, debiles in conatu ulcisci in ulla membra
Auxilii Christianorum, familias, amicos, communitates eorum, pro nobis
orantes familiasque illorum vel quicumque additus nobis et quibus sacerdotes
Auxilii Christianorum orarent. Ligamus omnes spiritus malignos, omnes
potentias in aëre, aquâ, terrâ, igne, sub terrâ vel ubicumque viribus suis
utantur, et ullam copiam satanicam in naturâ et ullos emissarios pretorii
satanici. Ligamus, in Pretiosissimo Sanguine Jesu, omnes attributa, idos,
vulticulos, adjuncta, interactiones ludusque dolosos spirituum malignorum.
Fregimus ulla et omnia ligamina, et adplicationes in nomine Patris, et Filii a
et Spiritus Sancti. Amen. 

\section{Feria V}

\lettrine{O}{mnipotens} Deus, Pater, te quæsumus pro redemptionem fratrum
sororumque quos a malo addictos esse per intercessionem et adjutorium
Archangelorum Sanctorum Michæl, Raphæl et Gabriel. Omnes Sancti Cœli,
ferte nobis auxilium.

{\RaggedRight
Ab anxietate, mæstitiâ et obsessionibus\\\tab -- \textit{Te rogamus libera nos Domine.}\\
Ab odio, fornicatione et invidiâ\\\tab -- \textit{Te rogamus libera nos Domine.}\\
\pagebreak
A cogitationibus zelotypiæ, furoris et mortis\\\tab -- \textit{Te rogamus libera nos Domine.}\\
Ab omni cogitatione suicidii et abortionis\\\tab -- \textit{Te rogamus libera nos Domine.}\\
Ab omni specie complexús peccabilis venerii\\\tab -- \textit{Te rogamus libera nos Domine.}\\
Ab omni divisione in familiâ nostrâ, et omni amicitate nocivâ\\\tab -- \textit{Te rogamus libera nos Domine.}\\
Ab omni genere incantationis, maleficii, artis magicæ et omni genere occulti\\\tab -- \textit{Te rogamus libera nos Domine.}
}
\ 

\noindent Da, qui dixisti “Pacem relinquo vobis, pacem meam do vobis”, per
intercessionem Beatae Mariæ Virginis, ab omni contagione dæmonica
libereamur et fruamur perenni pace tua. Per nomen Christi Domini nostri.
Amen. 

\section{Feria VI}

\subsection{Litania Humilitatis}

{\RaggedRight
\lettrine{O}\ Jesu, mitis et humilis corde,\\\tab\  \textit{exaudi me.}\\
A desiderio, ut amer,\\\tab \textit{libera me Domine.}\\
A desiderio, ut exalter,\\\tab \textit{libera me Domine.}\\
\pagebreak
A desiderio, ut honorer,\\\tab \textit{libera me Domine.}\\
A desiderio, ut lauder,\\\tab \textit{libera me Domine.}\\
A desiderio, ut aliis praeterear,\\\tab \textit{libera me Domine.}\\
A desiderio, ut consular,\\\tab \textit{libera me Domine.}\\
A desiderio, ut approber,\\\tab \textit{libera me Domine.}\\
A timore, ne humilier,\\\tab \textit{libera me Domine.}\\
A timore, ne spernar,\\\tab \textit{libera me Domine.}\\
A timore, ne contemnar,\\\tab \textit{libera me Domine.}\\
A timore, ne calumniam feram,\\\tab \textit{libera me Domine.}\\
A timore, ne oblivioni tradar,\\\tab \textit{libera me Domine.}\\
A timore, ne irridear,\\\tab \textit{libera me Domine.}\\
A timore, ne iniuriam accipiam,\\\tab \textit{libera me Domine.}\\
A timore, ne suspiciar,\\\tab \textit{libera me Domine.}\\
Ut magis alii amentur quam ego, Iesu,\\\tab \textit{da mihi gratiam ita desiderandi.}\\
\pagebreak
Ut plus alii aestimentur quam ego, Iesu,\\\tab \textit{da mihi gratiam ita desiderandi.}\\
Ut alii extollantur in mundi existimatione, ego autem minuar, Iesu,\\\tab \textit{da mihi gratiam ita desiderandi.}\\
Ut alii eligantur, ego autem praeterear, Iesu,\\\tab \textit{da mihi gratiam ita desiderandi.}\\
Ut alii mihi in omnibus rebus praeferantur, Iesu,\\\tab \textit{da mihi gratiam ita desiderandi.}\\
Ut alii me, dum sanctus fieri debeam, sanctiores sint, Iesu,\\\tab \textit{da mihi gratiam ita desiderandi.}\\
}

\section{Sabatto}
\lettrine{I}{nvoco} Nomen Sanctum tuum, Deus et Pater Domini Nostri Jesu Christi, et
clementiam tuam supplices deprecamur ut, nobis per intercessionem semper
Virginis Mariæ Matris, et Sancti Michælis Archangeli gloriosi, digneris
largiri auxilium contra satanam aliosque spiritus immundos qui pervagantur
in mundo magno cum periculo generis humani et damno animarum. Amen. 

\section{Conclusio Cotidiana}

\newcommand{\Kern}[2]{\addfontfeature{LetterSpace=#1}#2\addfontfeature{LetterSpace=0}\null}

\lettrine{A}{ugusta} Regina cœlorum, cœleste Superana Angelorum, quæ ab initio à Deo
recepisti potentiam et missionem ad caput satanæ conterendum, supplices
deprecamur te mittere tuas legiones sanctas, ut te imperante et potentiâ tuâ,
ubique persequantur demonia et confligant eos, deprimant audacitatem et
expellant eos in abysso. O bona et piissima Mater, semper amor et spes nostra
eris! O Mater divina, mitte Sanctos Angelos tuos ad nos defendendos et longe
hostem efferum expellendum. Sancti Angeli et Archangeli, defendite et
custodite nos. Amen.

\noindent Cor Jesu Sacratissimum, \textit{miserere nobis.}\\
Auxilium Christianorum, \textit{ora pro nobis.}\\
Virgo Potens, \textit{ora pro nobis.}\\
Sancte Joseph, \textit{ora pro nobis.}\\
Sancte Michæl Archangele, \textit{ora pro nobis.}\\
Omnes Sancti Angeli, \textit{orate pro nobis.}

\noindent In nomine Patris, \cross et Filii, et Spiritús Sancti. Amen

\vfill

\noindent \centering \cross Ad maiorem Dei gloriam \cross

\end{document}
